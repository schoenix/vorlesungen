\begin{algorithm}[H]
	\caption{Konvexe Hülle}
	\SetKwData{Left}{left}\SetKwData{This}{this}\SetKwData{Up}{up}
	\SetKwFunction{Union}{Union}\SetKwFunction{FindCompress}{FindCompress}
	\SetKwInOut{Input}{input}\SetKwInOut{Output}{output}

	\Input{Punktmenge $P=(p_1,..,p_n) \cap A^3$}
	\Output{$\lbrack P \rbrack$}
	\BlankLine

	\tcc{Verschiebe $P$, so dass der Ursprung innerhalb der Konvexen Hülle liegt}
	$v \longleftarrow 0 - \sum_i \frac{p_i}{n}$\newline
	$P \longleftarrow P + v$
	\BlankLine

	\tcc{Bestimme die Schnittmenge der ersten 4 HRe. Dadurch wird ein initialer Tetraeder aufgespannt}
	$V \longleftarrow$ Knotenliste von $p_1^{\leq} \cap ... \cap p_4^{\leq}$
	\BlankLine

	\tcc{Prüfe für alle weiteren HRe, ob diese in $V$ liegen}
	\For{$j > i$} {
		\eIf{$p_j^{\leq} \supset V$}{
			Entferne $p_j$ aus $P$\newline
			Aktualisiere "`gedanklich"' alle Indizes $k>j$
		}{
			Verknüpfe $p_j$ bidirektional mit einem $w_j \in V_4 \backslash p_j^{\leq}$
		}
	}
	\BlankLine

	\tcc{Schritt 3}
	\For{i = 5,...,n}{
		Durchlaufe $V$ von $w_i$ aus und setze dabei:\newline
			$N_i \longleftarrow \{ Schnittpunkte~der~Kanten~mit~p_i^{*}\}$\newline
			$W_i \longleftarrow V \cap p_i^{>}$\newline
			$V \longleftarrow (V \cup N_i) \backslash W_i$\newline
		\BlankLine
		\For{$j>i$ mit $w_j \in W_i$}{
			\eIf{$N_i \subset p_j^{\leq}$}{
				Entferne $P_j$ aus $P$\newline
				Aktualisiere "`gedanklich"' alle Indizes
			}{
				Verknüpfe $p_j$ neu mit einem $w_j \in N_i \cap p_j^{>}$
			}
		}
	}
	\BlankLine

	\Return{Verschobene Polarmenge $Q_P$}

\end{algorithm}